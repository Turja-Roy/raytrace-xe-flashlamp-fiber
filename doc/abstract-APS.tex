\documentclass[11pt]{article}
\usepackage[a4paper, margin=0.75in]{geometry}
\usepackage[document]{ragged2e}
\usepackage{graphicx}
\graphicspath{ {./images/} }
\usepackage{enumerate,enumitem}
\usepackage{framed}
\usepackage{amsmath,amsfonts,amsthm,thmtools,amssymb,mathtools,commath}
\usepackage{physics}
\usepackage{tikz}
\usetikzlibrary{mindmap}
\usepackage{caption}
\usepackage{xcolor}
\usepackage[most]{tcolorbox}
\usepackage{xparse}
\usepackage{hyperref,cleveref}
\usepackage{titlesec}
\usepackage{fancyhdr}
\usepackage{subcaption}
\usepackage{authblk}
\usepackage{cite}
\usepackage{xurl}
\usepackage{siunitx}
\sisetup{allow-number-unit-breaks=true}

\titlespacing\section{0pt}{8pt plus 0pt minus 2pt}{0pt plus 2pt minus 2pt}
\titlespacing\subsection{0em}{8pt plus 0pt minus 2pt}{0pt plus 2pt minus 2pt}
\titlespacing\subsubsection{0em}{8pt plus 0pt minus 2pt}{0pt plus 2pt minus 2pt}

\titleformat{\section}{\normalsize\bfseries}{\thesection}{1em}{}
\titleformat{\subsection}{\normalsize\bfseries}{\thesubsection}{1em}{}

\renewcommand\Authfont{\fontsize{11}{11.4}\selectfont}
\renewcommand\Affilfont{\fontsize{11}{11.4}\selectfont}

% \pagestyle{fancy}
% \fancyhf{}
% \lhead{Turja Roy}
% \rhead{PHYS-4281, Spring 2025}
% \rfoot{\thepage}


%%%%%%%%%%%%
%  Macros  %
%%%%%%%%%%%%
\newcommand{\R}{\mathbb{R}}
\newcommand{\N}{\mathbb{N}}
\newcommand{\Z}{\mathbb{Z}}
\newcommand{\Q}{\mathbb{Q}}
\newcommand{\C}{\mathbb{C}}
\newcommand{\F}{\mathbb{F}}
\newcommand{\E}{\mathbb{E}}
\newcommand{\B}{\mathcal{B}}
\newcommand{\U}{\mathcal{U}}
\newcommand{\V}{\mathcal{V}}
\newcommand{\W}{\mathcal{W}}
\newcommand{\X}{\mathcal{X}}
\newcommand{\Y}{\mathcal{Y}}
\newcommand{\A}{\mathcal{A}}
\newcommand{\D}{\mathcal{D}}
\newcommand{\I}{\mathcal{I}}
\newcommand{\J}{\mathcal{J}}
\newcommand{\K}{\mathcal{K}}
\newcommand{\M}{\mathcal{M}}
\newcommand{\T}{\mathcal{T}}
\newcommand{\G}{\mathcal{G}}
\newcommand{\HH}{\mathcal{H}}
\newcommand{\LL}{\mathcal{L}}
\newcommand{\PP}{\mathcal{P}}
\newcommand{\EE}{\mathcal{E}}

\usepackage{bm}

\title{\large\bfseries
    Systematic Optimization of Two-Lens VUV Coupling for LAr Purity Monitors
}

\author{Turja Roy}
\affil{Department of Physics, University of Texas at Arlington}
\date{}

\begin{document}

\twocolumn[
\begin{@twocolumnfalse}
    \maketitle

    \begin{abstract}
        An optimized two-lens plano-convex system is presented for coupling VUV light from a xenon flashlamp into optical fibers for liquid argon purity monitoring. Light from a highly divergent 3 mm arc source (66-degree) is coupled into a 1 mm fiber (0.22 NA), representing a challenging design constraint. Six optimization algorithms are compared using ray tracing with geometric optics and atmospheric absorption: grid search, Powell's method, differential evolution, Nelder-Mead, dual annealing, and Bayesian optimization. Efficiency is evaluated propagating 1000 rays using stratified Monte Carlo sampling. Atmospheric absorption by O2, N2, and H2O is modeled empirically. Optimized configurations achieve efficiencies of 0.19-0.24 in air and 0.24-0.27 in argon, primarily limited by source divergence; reduced divergence would significantly improve coupling. An 8\% improvement is observed in argon due to eliminated O2 absorption. System lengths are 35-41 mm. Powell's method provides efficient routine optimization (9 s), while differential evolution offers thorough global search (49 s). Wavelength analysis across 150-300 nm shows peak performance at 220-260 nm, with severe degradation below 200 nm in air. The framework is generalizable to other lens types, gas media, and wavelength regimes. Future work includes tolerance analysis and experimental validation. These results provide design guidelines for VUV coupling systems in experimental physics.
    \end{abstract}
\end{@twocolumnfalse}
]

\end{document}
